\chapter{Model Overview}
\section{Introduction}
This section details the governing equations and numerical methods used to simulate the \gls{sco2} jet. As introduced previously, the compressible Navier-Stokes equations form the basis of the model along with the \gls{srk}. Here we also detail the filtering used in the \gls{les} and the \gls{smd} used to model the \gls{sgs} dynamics \cite{LES_Comp}. 

To discretize and evolve the system of partial differential equations, including the \gls{les} \gls{sgs} terms, we use \textit{PeleC}, a compressible hydrodynamics code for reacting flows that leverages \textit{AMReX} for \gls{amr} \cite{PeleC1, PeleC2}. \textit{PeleC} is a highly scalable code for heterogeneous architectures that is being developed as part of the \gls{ecp} through the \gls{doe}. It leverages the \textit{PelePhysics} library for complex physics, including chemical reactions, non-ideal \gls{eos}, and high fidelity transport models. For spatial discretization, \textit{PeleC} contains a few variations of the general \gls{ppm} originally derived by Colella and Woodward \cite{1984JCoPPPM}. We utilize a variation that allows for extrema preservation in the presence of steep gradients \cite{MILLER200226, COLELLA20087069}. For time discretization, a second order Runge-Kutta scheme is used, and the time step is dynamically limited using a Courant number of 0.9. 
\section{Governing Equations}

We consider the three-dimensional compressible Navier-Stokes equations, presented here with Einstein notation:
\begin{subequations} \label{NSE_einstein}
\begin{align}
  \frac{\partial\rho}{\partial t} + \frac{\partial }{\partial x_j} \left( \rho u_j \right) &= 0,  \label{NSE_mass}\\
  \frac{\partial}{\partial t} \left( \rho u_i \right) + \frac{\partial}{\partial x_j} \left(\rho u_i u_j + p \delta_{ij} -\sigma_{ij} \right) &= 0,  \label{NSE_mom}\\
  \frac{\partial}{\partial t} \left( \rho E \right) + \frac{\partial}{\partial x_j} \big(\left( \rho E+p \right) u_j + q_j - \sigma_{ij} u_i\big) &= 0 \label{NSE_E}
\end{align}
\end{subequations}
where $\rho$ is the density, $u_j$ is the velocity for the $x_j$ direction, $p$ is the pressure, $E = e + \frac{u_i u_i}{2}$ is the total energy, $e $ is the internal energy, and $T$ is the temperature. Following the assumptions made for Newtonian fluids \cite{}, the diffusive fluxes are
\begin{equation} \label{Transport}
  \sigma_{ij} = 2\mu S_{ij} - \frac{2}{3}\mu \delta_{ij} S_{kk}, \quad \quad
  q_j = -\lambda \frac{\partial T}{\partial x_j}
\end{equation}
where
$S_{ij} = \frac{1}{2}\left(\frac{\partial u_i}{\partial x_j} + \frac{\partial u_j}{\partial x_i} \right)$ is the strain-rate tensor, $\mu$ is the dynamic viscosity, and $\lambda$ is the thermal conductivity. Models regarding these two components are given in more detail in the next section. $\delta_{ij}$ here is the Kronecker delta. External forces such as gravity are not included in this study. The system is closed using the \gls{srk} \cite{SOAVE1972} to relate pressure, density, and temperature as follows:
\begin{equation} \label{SRK_eos}
\begin{aligned} 
	p &= \dfrac{RT}{V_m - b} - \dfrac{a \alpha}{V_m(V_m + b)}, \\
	a &= \dfrac{0.42747R^2T_c^2}{P_c}, \\
	b &= \dfrac{0.08664RT_c}{P_c}, \\
	\alpha &= \left( 1 + \left( 0.48508 + 1.55171 \omega - 0.15613 \omega^2 \right)\left( 1 - T_r^{0.5} \right) \right)^2
\end{aligned}
\end{equation}
where $R$ is the ideal gas constant, $T_c$ and $P_c$ are the critical temperature and pressure of the species, respectively, $T_r = T/T_c$ is the reduced temperature given by the ratio of the absolute temperature to the critical temperature, $V_m$ is the molar volume of the species, and $\omega$ is the acentric factor of the species. All cases are run with a single species, that being \gls{co2}. 

\subsection{Transport Models}
Modeling of transport coefficients is done through \textit{PelePhysics} \cite{}. There are three modeling options available in \textit{PelePhysics}; we use the \textit{Simple} model. The \textit{Simple} model approximates ideal gas transport coefficients using EGLib functions \cite{ERN1995105}, which have the following form:
\begin{equation} \label{EGLib}
\ln{(q_0)} = \sum\limits_{n=1}^{4} a_{q,n}\left( \ln{(T)}\right)^{n-1}
\end{equation}
where $q_0$ is the transport quantity of interest (either $\lambda$ or $\mu$) and $a$ is the appropriate pre-calculated polynomial fit coefficient (see Appendix \ref{}). Chung's high pressure correction for viscosity and thermal conductivity are included to account for real gas dynamics \cite{chung:1988}: 
\begin{equation} \label{chung_general}
q = q_k + q_p
\end{equation}
where $q_k$ is the low-pressure gas transport quantity related to the ideal gas quantity $q_0$ and $q_p$ is the high-pressure deviation. 

For viscosity, these quantities are:
\begin{equation}
\begin{split}
\mu_k = \mu_0 \left( \dfrac{1}{G_2} + A_6 Y \right), \\ 
\mu_p = \left(\dfrac{\num{36.344e-6}(MT_c)^{1/2}}{V_c^{2/3}}\right)A_7Y^2G_2\exp(A_8 + \dfrac{A_9}{T^*} + \dfrac{A_{10}}{T^{*2}}) 
\end{split}
\end{equation}
where $M$ is the molecular weight, $V_c$ is the critical molar volume, $T^* = T/\epsilon_k$ is a dimensionless temperature scaling using the Lennard-Jones potential well depth \cite{}, $Y = (\rho V_c)/6$, and $G_1 = (1-0.5Y)/(1-Y)^3$, and $G_2 = \left\{A_1\left[   1-\exp(-A_4Y)\right]/Y + A_2G_1\exp(A_5Y) + A_3G_1 \right\}/(A_1A_4 + A_2 + A_3)$. The constants $A_{1-10}$ are linear functions calculated as follows: 
\begin{equation}
A_i = a_{i0} + a_{i1} \omega + a_{i2} \mu_r^4 + a_{i3} \kappa \quad i = 1,..., 10
\end{equation} 
where $\mu_r$ is the reduced dipole moment of the species, $\kappa$ is the association factor of the species, and $a_{ij}$ are constants (see Appendix \ref{chung_viscosity_aij}).  

Similarly, thermal conductivity components are given by: 
\begin{equation}
\begin{split}
\lambda_k = \lambda_0 \left( \dfrac{1}{H_2} + B_6 Y \right), \\ 
\lambda_p = \left(\dfrac{\num{3.039e-4}(T_c/M)^{1/2}}{V_c^{2/3}}\right)B_7Y^2H_2T_r^{1/2} 
\end{split}
\end{equation}
where $H_2 = \left\{B_1\left[   1-\exp(-B_4Y)\right]/Y + B_2G_1\exp(B_5Y) + B_3G_1 \right\}/(B_1B_4 + B_2 + B_3)$ and $B_{1-7}$ are defined as: 
\begin{equation}
B_i = b_{i0} + b_{i1} \omega + b_{i2} \mu_r^4 + b_{i3} \kappa \quad i = 1,..., 7
\end{equation}
where $b_{ij}$ are constants (see Appendix \ref{chung_conductivity_bij}). All species-related constants mentioned in this section can also be found in Table \ref{}. 

These high pressure corrections as derived and analyzed by Chung et al. show improved accuracy for approximating transport properties over a wide range of temperatures and pressures when compared to experimental data for a variety of species \cite{chung:1988}. For carbon dioxide in particular,  

\subsection{Thermodynamics and Related Quantities}
Modeling of thermodynamic quantities is also done through \textit{PelePhysics} \cite{}. In a similar fashion to equation \ref{chung_general}, thermodynamic properties can be broken up into an ideal gas component and a departure from ideal gas behavior:
\begin{equation} \label{thermo_general}
q(\phi) = q_{I}(\phi) + q_{D}(\phi)
\end{equation}
where $\phi$ is the appropriate combination of state variables involving temperature, specific volume, and pressure, $q_{I}$ is the ideal state component and $q_{D}$ is the departure component. These decompositions can be derived from fundamental thermodynamic relations \cite{}. For example, change in internal energy can be written as follows:
\begin{equation} \label{thermo_example}
de = \left( \dfrac{\partial e}{\partial T} \right)_{V} dT + \left[ T \left( \dfrac{\partial p}{\partial T} \right)_{V} - p  \right] dV
\end{equation}
substituting in the appropriate partial derivative using the \gls{srk} and integrating from an ideal state reference point $(v,t)$ to general point far from ideal conditions $(V,T)$ yields:
\begin{equation}  \label{thermo_SRK_sub}
e(V,T) - e(v,t) =  e(v,T) - e(v,t) + a \left[ \alpha - T \dfrac{\partial \alpha}{\partial T} \right] \int\limits_{v}^{V} \dfrac{1}{V_m(V_m + b)}dV'
\end{equation}
Simplifying the left- and right-hand side of Equation \ref{thermo_SRK_sub} and utilizing the choice of $v \to \infty$ for the ideal state volume \cite{}, we get:
\begin{equation} \label{thermo_simplify}
e(V,T) =  e(\infty,T) + a \left[ \alpha - T \dfrac{\partial \alpha}{\partial T} \right] \int\limits_{\infty}^{V} \dfrac{1}{V_m(V_m + b)}dV'
\end{equation}
Evaluating the integral in Equation \ref{thermo_simplify}, we arrive at the final expression for the internal energy of the species described by the \gls{srk}: 
\begin{equation} \label{thermo_final}
e(V,T) =  e_I(T) + a \left[ T \dfrac{\partial \alpha}{\partial T} - \alpha \right] \dfrac{1}{b}\ln\left( 1 + \dfrac{b}{V_m} \right)
\end{equation}
where $e_I(T)=c_vT$ is the ideal internal energy which is only dependent on temperature and $e_D(T,V) = a \left[ T \tfrac{\partial \alpha}{\partial T} - \alpha \right] \tfrac{1}{b}\ln\left( 1 + \tfrac{b}{V_m} \right)$ is the departure function for the internal energy. A similar procedure can be done to arrive at the decomposed enthalpy equation involving \gls{srk}-specific quantities:
\begin{equation} \label{thermo_enthalpy}
h(T,V) = h_I(T) +  a \left[ T \dfrac{\partial \alpha}{\partial T} - \alpha \right] \dfrac{1}{b}\ln\left( 1 + \dfrac{b}{V_m} \right) + \dfrac{RTb}{V_m - b} - \dfrac{a \alpha}{V_m + b}
\end{equation}
The departure function can be directly evaluated while the ideal gas portion must be approximated.

Ideal specific heat at constant volume scaled by the ideal gas constant $\nicefrac{c_{v_I}}{R}$ is approximated with NASA polynomial fits \cite{NASAPoly}. These polynomial fits have the following form:
\begin{equation} \label{NASA_poly}
\psi(T) = c_{\psi, 0} + c_{\psi, 1}T + c_{\psi, 2}T^2 + c_{\psi, 3}T^3 + c_{\psi, 4}T^4  + c_{\psi, 5}\dfrac{1}{T}
\end{equation}
where $c_{\psi,0-5}$ are species-specific constants for each quantity's polynomial fit (see Appendix \ref{}). Other ideal gas quantities of interest can also be calculated in this form via integration, so that $\psi$ here can be either ideal internal energy $\nicefrac{e_I}{RT}$, specific heat at constant volume $\nicefrac{c_{v_I}}{R}$, or enthalpy $\nicefrac{h_I}{RT}$. 
The equations for additional thermodynamic quantities, such as constant volume and constant pressure specific heat, and useful derivatives involving the \gls{srk} can be found in Appendix \ref{}. 

\subsection{Filtered Navier-Stokes Equations}
To perform the \gls{les}, we consider the filtered compressible Navier-Stokes equations as implemented by Mart\'{i}n, Piomelli, and Graham \cite{LES_Comp}. Here we go through a brief derivation along with the main assumptions needed. First, applying the filtering operation from Equation \ref{les_filtering} to Equations \ref{NSE_einstein}, we get:
\begin{subequations} \label{NSE_filtered}
\begin{align}
  \frac{\partial\overline{\rho}}{\partial t} + \frac{\partial }{\partial x_j} \left( \overline{\rho u_j} \right) &= 0,  \label{mass_filtered} \\
  \frac{\partial}{\partial t} \left( \overline{\rho u_i }\right) + \frac{\partial}{\partial x_j} \left(\overline{\rho u_i u_j} + \overline{p }\delta_{ij} - \overline{\sigma_{ij}} \right) &= 0,  \label{mom_filtered} \\
  \frac{\partial}{\partial t} \left( \overline{\rho E} \right) + \frac{\partial}{\partial x_j} \big(\left( \overline{\rho Eu_j}+\overline{p u_j} \right) + \overline{q_j} - \overline{\sigma_{ij} u_i}\big) &= 0 \label{en_filtered}
\end{align}
\end{subequations}
In order to avoid having to model $\overline{\rho u_j} $ in the conservation of mass equation in \ref{NSE_filtered}, Favre-filtering, $\widetilde{\cdot}  = \nicefrac{\overline{\rho~\cdot}}{\overline{\rho}}$, is also applied \cite{}. Additionally, the following two sets of assumptions are made regarding transport terms \cite{}:
\begin{subequations} \label{trans_assumptions}
\begin{align}
\overline{\mu(T)S_{ij}} \simeq \mu(\widetilde{T})\widetilde{S_{ij}}, \quad \quad 
\overline{\lambda(T)\dfrac{\partial T}{\partial x_j}} \simeq \lambda(\widetilde{T})\dfrac{ \partial \widetilde{T}}{\partial x_j} \label{trans_1}\\
\widetilde{\mu} = \mu(\widetilde{T}), \quad \quad \widetilde{\lambda} = \lambda(\widetilde{T}) \label{trans_2}
\end{align}
\end{subequations}
The assumptions in Equations \ref{trans_assumptions} are common in the literature regarding \gls{les}, even though transport terms are nonlinearly dependent on temperature \cite{}. Justification for this type of assumption can be made where if performing a \gls{les} with course enough grid, molecular transport coefficients remain small compared to turbulent transport coefficients \cite{}. Applying Equations \ref{trans_assumptions} along with the Favre filter to Equations \ref{NSE_filtered} yields the following:
\begin{equation} \label{NSE_favre_filtered}
\begin{split}
  \frac{\partial\overline{\rho}}{\partial t} + \frac{\partial }{\partial x_j} \left( \overline{\rho}\widetilde{ u_j} \right) = 0,  \\
  \frac{\partial}{\partial t} \left( \overline{\rho}\widetilde{ u_i }\right) + \frac{\partial}{\partial x_j} \left(\overline{\rho}\widetilde{ u_i u_j} + \overline{p }\delta_{ij} - \widetilde{\sigma_{ij}} \right) = 0,  \\
  \frac{\partial}{\partial t} \left( \overline{\rho}\widetilde{ E} \right) + \frac{\partial}{\partial x_j} \left(\left( \overline{\rho}\widetilde{ Eu_j}+\overline{p u_j} \right) + \widetilde{q_j} - \overline{\sigma_{ij} u_i}\right) = 0 
\end{split}
\end{equation}
with filtered Equations \ref{Transport} now given by: 
\begin{equation} \label{filtered_trans}
  \widetilde{\sigma_{ij}} = 2\widetilde{\mu}\widetilde{ S_{ij}} - \frac{2}{3}\widetilde{\mu} \delta_{ij} \widetilde{ S_{kk}}, \quad \quad
  \widetilde{q_j} = -\widetilde{\lambda} \frac{\partial \widetilde{T}}{\partial x_j}
\end{equation}

Final simplifications to Equations \ref{NSE_favre_filtered} come from the \gls{sgs} terms. The \gls{sgs} stress $\tau_{ij}$, \gls{sgs} heat flux $\mathcal{Q}_j$, \gls{sgs} turbulent diffusion $\nicefrac{\partial \mathcal{J}_j}{\partial x_j}$, and \gls{sgs} turbulent viscous diffusion $\nicefrac{\partial \mathcal{D}_j}{\partial x_j}$ set through the following definitions \cite{}:
\begin{equation} \label{sgs_defs}
\begin{aligned}
\tau_{ij} &= \overline{\rho}\left(\widetilde{u_i u_j} - \widetilde{u_i}\widetilde{u_j} \right), \\
\mathcal{Q}_{j} &= \overline{\rho}\left(\widetilde{u_j T} - \widetilde{u_j}\widetilde{T} \right), \\
\mathcal{J}_{j} &= \overline{\rho}\left(\widetilde{u_j u_k u_k} - \widetilde{u_j}\widetilde{u_k u_k} \right), \\
\mathcal{D}_{j} &= \overline{\rho}\left(\widetilde{\sigma_{ij}u_i} - \widetilde{\sigma_{ij}}\widetilde{u_i} \right)
\end{aligned}
\end{equation} 
After substituting the appropriate pieces of Equations \ref{sgs_defs} into the momentum and energy components of \ref{NSE_favre_filtered} and applying a few further assumptions, we can eventually get the following:
\begin{subequations}
\begin{align}
  \frac{\partial\overline{\rho}}{\partial t} + \frac{\partial }{\partial x_j} \left( \overline{\rho}\widetilde{ u_j} \right) = 0, \label{NSE_mass_sgs} \\
  \frac{\partial}{\partial t} \left( \overline{\rho}\widetilde{ u_i }\right) + \frac{\partial}{\partial x_j} \left(\overline{\rho}\widetilde{ u_i} \widetilde{u_j} + \overline{p }\delta_{ij} - \widetilde{\sigma_{ij}} \right) = - \frac{\partial \tau_{ij}}{\partial x_j}, \label{NSE_mom_sgs}  \\
  \frac{\partial}{\partial t} \left( \overline{\rho}\widetilde{ E} \right) + \frac{\partial}{\partial x_j} \left(\left( \overline{\rho}\widetilde{ E}+\overline{p} \right)\widetilde{u_j} + \widetilde{q_j} - \widetilde{\sigma_{ij}}\widetilde{ u_i}\right) = - \frac{\partial}{\partial x_j } \left( \gamma c_v \mathcal{Q}_j + \dfrac{1}{2} \mathcal{J}_j - \mathcal{D}_j \right)  \label{NSE_E_sgs}
\end{align}
\end{subequations}
with $\gamma = \nicefrac{c_p}{c_v}$. The simplification above hinges upon relating enthalpy to internal energy and pressure as in the ideal gas case with $h = e ~+~ \nicefrac{p}{\rho} = c_p T$ with $c_p$ assumed to be constant. Neither of these assumptions is consistent with supercritical fluids but is chosen here for ease of splitting Equations \ref{} into simulated vs. modeled quantities. This is also standard practice in many supercritical \gls{les} studies \cite{}, though these assumptions leave room for future work on incorporating higher order \gls{eos} into the \gls{les} derivation. 

Finally, it is also noted that the divergence of the \gls{sgs} heat flux and the \gls{sgs} turbulent diffusion are of comparable order of magnitude while the \gls{sgs} viscous diffusion is an order of magnitude smaller \cite{LES_Comp}. Therefore, the \gls{sgs} turbulent viscous diffusion is omitted from the system. The final set of equations to be discretized and advanced is given by:  
\begin{subequations} \label{filtered_NSE_FINAL}
\begin{align}
  \frac{\partial\overline{\rho}}{\partial t} + \frac{\partial }{\partial x_j} \left( \overline{\rho}\widetilde{ u_j} \right) = 0, \label{NSE_mass_FINAL} \\
  \frac{\partial}{\partial t} \left( \overline{\rho}\widetilde{ u_i }\right) + \frac{\partial}{\partial x_j} \left(\overline{\rho}\widetilde{ u_i} \widetilde{u_j} + \overline{p }\delta_{ij} - \widetilde{\sigma_{ij}} \right) = - \frac{\partial \tau_{ij}}{\partial x_j}, \label{NSE_mom_FINAL}  \\
  \frac{\partial}{\partial t} \left( \overline{\rho}\widetilde{ E} \right) + \frac{\partial}{\partial x_j} \left(\left( \overline{\rho}\widetilde{ E}+\overline{p} \right)\widetilde{u_j} + \widetilde{q_j} - \widetilde{\sigma_{ij}}\widetilde{ u_i}\right) = - \frac{\partial}{\partial x_j } \left( \gamma c_v \mathcal{Q}_j + \dfrac{1}{2} \mathcal{J}_j \right)  \label{NSE_E_FINAL}
\end{align}
\end{subequations}
with Equations \ref{filtered_trans} describing the filtered transport coefficients. The \gls{sgs} stress $\tau_{ij}$, \gls{sgs} heat flux $\mathcal{Q}_{j}$, and \gls{sgs} turbulent diffusion $\mathcal{J}_j$ need to be modeled in order to close the system. 



\subsection{Subgrid-Scale Modeling for Large Eddy Simulation}
We use the \gls{smd} \gls{les} model for compressible flow as described by Mart\'{i}n, Piomelli, and Graham \cite{LES_Comp}. In this work, the grid provides the implicit filtering of the equations. The \gls{sgs} stress tensor, $\tau_{ij}$, is included in the diffusive fluxes and is calculated as follows:
\begin{equation}
\begin{aligned}
	\tau_{ij} - \dfrac{\delta_{ij}}{3}\tau_{kk} &= -C_s^22\overline{\Delta}^2 \overline{\rho} |\widetilde{S}| \left( \widetilde{S}_{ij} - \dfrac{\delta_{ij}}{3} \widetilde{S}_{kk} \right) = C_s^2 \alpha_{ij},  \\
	 \tau_{kk} &= C_I 2\overline{\rho} \overline{\Delta}^2 |\widetilde{S}|^2 = C_I \alpha
\end{aligned}
\end{equation}
with $\overline{\Delta}$ being the filter width associated with the smallest scale retained by the filtering operation ($\overline{\Delta}$ is the grid spacing for our cases). Additionally, $|\widetilde{S}| = (2\widetilde{S}_{ij}\widetilde{S}_{ij})^{1/2}$. The two model coefficients are calculated as follows:
\begin{equation} \label{smd_coeffs}
\begin{aligned}
	C = C_s^2 = \dfrac{\langle \mathcal{L}_{ij} M_{ij} \rangle}{\langle M_{kl}M_{kl} \rangle}, \quad \quad C_I = \dfrac{ \langle \mathcal{L}_{kk} \rangle}{\langle \beta - \widehat{\alpha} \rangle}
\end{aligned}
\end{equation}
where the Germano identity, $\mathcal{L}_{ij} = T_{ij} - \widehat{\tau}_{ij}$, is used to relate the \gls{sgs} stress tensor to the ``resolved turbulent stresses", $\mathcal{L}_{ij} = \left( \overline{\rho u_i} \widehat{ \hspace{1.5pt} \overline{\rho u_j}}/\overline{\rho} \right) - \widehat{ \overline{\rho u_i}} \hspace{1.5pt} \widehat{\overline{\rho u_j}}/\widehat{\overline{\rho}}$ ,and the subtest stresses, $T_{ij} = \widehat{\overline{\rho}} \breve{\widetilde{u_i u_j}} - \widehat{\overline{\rho}} \breve{\widetilde{u_i}} \breve{\widetilde{u_j}}$ \cite{germano}. In this relationship, a hat denotes quantities associated with a test filter $\widehat{G}$ which has a characteristic length of $\widehat{\Delta}$. The breve denotes Favre-filtered quantities using $\widehat{G}$ (i.e., $\breve{\widetilde{f}} = \nicefrac{\widehat{\overline{\rho f}}}{\widehat{\overline{\rho}}}$). Additionally, $M_{ij} = \beta_{ij} - \widehat{\alpha_{ij}}$ with $\beta_{ij} = -2\widehat{\Delta}^2 \widehat{\overline{\rho}} |\breve{\widetilde{S}}| \left( \breve{\widetilde{S}}_{ij} - \delta_{ij} \breve{\widetilde{S}}_{kk}/3  \right)$ and $\beta = 2 \widehat{\Delta}^2  \widehat{\overline{\rho}} |\breve{\widetilde{S}}|^2 $. 


The \gls{sgs} heat flux $\mathcal{Q}_j$ is also modeled dynamically as in \cite{LES_Comp}:
\begin{equation} \label{sgs_heat_flux}
\mathcal{Q}_j = - \dfrac{\overline{\rho}\nu_T}{Pr_T} \dfrac{\partial \widetilde{T}}{\partial x_j} = - C\dfrac{\overline{\Delta}^2 \overline{\rho}  |\widetilde{S}|}{Pr_T} \dfrac{\partial \widetilde{T}}{\partial x_j} 
\end{equation}
where $C$ is modeled as in Equation \ref{smd_coeffs} and the turbulent Prandtl number, $Pr_\text{T}$, is calculated dynamically as:
\begin{equation}
\begin{aligned}
	Pr_\text{T} = \dfrac{C \langle T_k T_k  \rangle}{\langle \mathcal{K}_j T_j \rangle}
\end{aligned}
\end{equation}
where 
\begin{equation}
\begin{aligned}
	T_j = - \widehat{\Delta}^2  \widehat{\overline{\rho}} |\breve{\widetilde{S}}| \dfrac{\partial \breve{\widetilde{T}}}{\partial x_j} +  \overline{\Delta}^2  \overline{\rho} \widehat{|\widetilde{S}| \dfrac{\partial \widetilde{T}}{\partial x_j}}, \qquad \qquad \mathcal{K}_j = \left( \dfrac{ \widehat{\overline{\rho u_j} \overline{\rho T}}}{\overline{\rho}} \right) - \dfrac{ \widehat{\overline{\rho u_j}} \widehat{ \overline{\rho T}}}{\widehat{\overline{\rho}}}.
\end{aligned}
\end{equation}

Finally, the \gls{sgs} turbulent diffusion $\mathcal{J}_j$ is modeled following the strategy proposed by Knight et al. \cite{}:
\begin{equation} \label{sgs_turb_diff}
\mathcal{J}_j = \widetilde{u}_k \tau_{jk}
\end{equation}

For our simulations, we implement the three point box filter as described in \cite{filter} with a filter-grid ratio of 2, i.e.\  $\widehat{\Delta}=2\overline{\Delta}$. With this filter, the convolution kernel from Equation \ref{les_filtering} is defined as follows: 
\begin{equation}
G(\vb{x} - \vb{r}) = 
\begin{cases} 
      \dfrac{1}{\widehat{\Delta}} & |\vb{x} - \vb{r}|\leq \dfrac{\widehat{\Delta}}{2}, \vspace{2pt}\\ 
      0 & \text{otherwise}
   \end{cases}
\end{equation}
The discrete implementation of this filtering function is described in the numerical methods section. 

As noted previously, the choice of \gls{sgs} modeling is important in accurately capturing the turbulence statistics of the system. M\"{u}ller et al. found that while the choice in thermodynamic modeling is crucial in capturing first-order moments, the effects \gls{sgs} modeling is limited to second-order moments \cite{doi:10.1063/1.4937948}. Therefore, our conclusions relating to the turbulence dynamics will be unaffected. That being said, \cite{doi:10.1063/1.4937948} also notes that the choice in \gls{sgs} model and numerical flux discretization had a larger than expected effect on resolved Reynolds stress profiles. Specifically, the constant Smagorinsky model yielded decaying fluctuation magnitudes during early evolution, resulting in the transition to a fully turbulent mixing zone to start from lower turbulence levels. However, this did agree with the jet break-up location inferred from mean density profiles, which were shifted slightly downstream by comparison to other \gls{sgs} models. These relationships will be taken into consideration for this study as well and noted in the discussion of Reynolds stress profiles.



\section{Validation}
Make plots to compare with NIST data.


