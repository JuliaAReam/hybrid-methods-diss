\chapter{Introduction}
This chapter provides an overview of the important mathematics concepts and fluid properties relevant to the work presented in this project. Additionally, this background provides the context for which this work exists and the contributions made through it to the field of computational fluid dynamics. First, the mathematical framework for modeling compressible Newtonian fluids is provided to form the basis of the modeling done in this dissertation. Further consideration is then given to turbulence modeling and the numerical methods developed for studying turbulence to provide insight into the quantities of interest analyzed within this dissertation and the choices of numerical methods used herein. Supercritical fluids are then defined and the challenges associated with studying them are presented. Important applications of supercritical carbon dioxide in particular are provided to motivate the problem presented in this dissertation. Existing numerical studies on supercritical fluids are reviewed to demonstrate how this dissertation fits into the current landscape of research and to emphasize the contribution the results of this work make to the field. This chapter concludes with an outline of the dissertation, the goals of the dissertation, and the main contributions made through this work. 

\section{Mathematical Description of Fluid Flow}
We interact with fluids everyday as a part of the human experience, with familiar examples being found in the water we drink and the breeze felt on a warm summer day. From that perspective, the average person would most likely define a fluid as ``something that flows" (this was at least the general consensus reached by the class on our first day of ``Introduction to Fluid Dynamics" when posed with the task.) In order to mathematically describe these substances, it is helpful to have a more concrete idea of what it means to be a fluid. 

A \textit{fluid} is a large collection of mutually interacting particles (e.g. molecules, atoms, etc.) in a state of constant and chaotic motion. This results in the continuous deformation of the substance under the effects of a shearing stress. From a kinematics perspective, particle motion within a fluid can be broken up into two phases: particle interaction and free flight. Average time spent in free flight, $t_f$, is typically much greater than collision time at an instant of interaction, $t_c$. Free flight  

\subsection{Navier-Stokes Equations}
\subsection{Equations of State}

\section{Turbulence}
Additionally, fluid flow can be categorized into different types based on certain defining flow characteristics. The main two classifications of note are laminar flow and turbulent flow.

Laminar flow is denoted by fluid particles having well-defined parallel trajectories of motion, or streamlines. Streamlines do not cross, meaning adjacent layers within the fluid flow by one another with little to no mixing. From a more generalized perspective, the flow appears to be smooth. In contrast to this, turbulent flow is characterized by its unpredictable and chaotic trajectories. Streamlines do cross resulting in swirls and eddies of varying length scales which induce mixing. Turbulent flow can be qualitatively described as being rough due to this high degree of fluctuation within the velocity and pressure fields present. This generalized description is depicted in Figure \ref{lam_vs_turb}.

\begin{figure}[h!]
\begin{center}
\includegraphics[scale=0.5]{figures/laminar}
\includegraphics[scale=0.503]{figures/turbulent}
\end{center}
\caption{Example of streamlines in laminar (left) vs. turbulent (right) flow.}
\label{lam_vs_turb}
\end{figure}

The Reynolds number is a dimensionless value that can be used to distinguish laminar flow from turbulent flow. It is defined as follows:
\begin{equation}
\label{Reynolds_num}
\text{Re} = \frac{\rho u L}{\mu}
\end{equation}

\noindent where $\rho$ and $\mu$ are the density and dynamic viscosity of the fluid, respectively, $u$ is the flow velocity, and $L$ is a characteristic length scale associated with the given flow scenario (e.g. pipe diameter). As is demonstrated by the ratio in Equation \ref{Reynolds_num}, the Reynolds number measures the relative effects of inertial forces compared to viscous forces within a given flow scenario. A small Reynolds number signifies the dominance of viscous forces; fluid parcels moving in tandem want to ``stick together," resulting in the sheared flow and parallel trajectories seen in laminar flow. Turbulence is then characterized by a large Reynolds number, where inertial forces take precedence. Here, deviations within the laminar flow field result in lateral mixing between shear layers. This creates eddies and random trajectories that result in the chaotic motion of turbulent flow. 

This work focuses on the turbulent round jet and its associated dynamics in the context of supercritical fluids. The remainder of this section details a brief historical overview of turbulence modeling and numerical methods developed for studying turbulence in order to motivate the modeling and numerical choices made within this work.

\subsection{Historical Perspective}


\subsection{Numerical Approaches}


\section{Supercritical Fluids}

\subsection{Applications of Supercritical Carbon Dioxide}
Fluids also serve an important role in industry, proving useful in areas ranging from thermal management to energy production; steam turbines alone accounted for 45\% of electricity generation in the United States in 2021 \cite{US_elec_gen_stat}.  
\subsection{Overview of Current Numerical Landscape}






