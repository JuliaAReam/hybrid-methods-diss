\chapter{Conclusion}

In this work, we studied three cases of a supercritical round turbulent jet injected into a supercritical environment: an isothermal case, a non-isothermal case with ambient temperature further away from the critical point, and a non-isothermal case with ambient temperature closer to the critical point and specifically crossing a region of intense thermodynamic fluctuation. It was shown that the isothermal case had many similarities with ideal incompressible round jets. Many properties found in classical round turbulent jets were recovered in this case, such as self-similarity, linear decay along the centerline, and general trends associated with Reynolds stresses (although self-similarity in the Reynolds stresses was not recovered; general trends were still agreeable). The non-isothermal case further away from the critical point behaved similarly to the isothermal case with slight differences being noted in potential core length and spreading of the jet. The non-isothermal case that transited the pseudo-boiling point exhibited noticeably different behavior. These trends were compared with previous studies involving transcritical injection to see how phase lack of full phase change impacted flow characteristics. 

Some limitations of this work are now noted. \gls{sgs} models do not take into consideration real gas behavior, being limited to assumptions based on the ideal gas equation of state. There is room for improvement in turbulence modeling with the aim of incorporating higher order equations of state. As with any numerical simulation, discretization and resolution inherently lead to error. Ideally with more power, finer resolution can be achieved to help improve simulations. Finally, minor artifacts were noted in the non-isothermal case closest to the critical point due to the presence of steep gradients. Though the impact of these are believed to be mild, they are still present and add to cumulative error. 

Future work aims to incorporate multiple species to the flow system and to investigate higher Mach flows. More specific application configurations can also be analyzed via incorporation of complex geometries. 


