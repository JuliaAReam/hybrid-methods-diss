\chapter{Conclusion}

In this work, we studied three cases of a supercritical round turbulent jet injected into a supercritical environment: an isothermal case, a non-isothermal case with ambient temperature further away from the critical point, and a non-isothermal case with ambient temperature closer to the critical point and specifically crossing a region of intense thermodynamic fluctuation known as the pseudo-boiling point. It was shown that the isothermal case had many similarities with ideal incompressible round jets. Many properties found in classical round turbulent jets were recovered in this case, such as self-similarity, linear decay along the centerline, and general trends associated with Reynolds stresses (although self-similarity in the Reynolds stresses was not recovered; general trends were still agreeable). The non-isothermal case further away from the critical point behaved similarly to the isothermal case with slight differences being noted in potential core length and spreading rate of the jet. The non-isothermal case that transited the pseudo-boiling point exhibited noticeably different behavior.

The effects of pseudo-boiling density stratification was demonstrated for low-to-high density supercritical jet injection, in contrast to the more common high-to-low density injection scheme found in existing literature. Earlier onset of Kelvin-Helmholtz-like instabilities in the jet-ambient interface ultimately contributes to faster mixing and jet decay as compared to the other non-isothermal case presented here. Of note with this regard, resolved turbulent kinetic energy in this case is redirected from the spanwise direction to the streamwise direction, displaying enhanced isotropy in the pseudo-boiling case. To the author's knowledge, this feature has not been explored in the current literature regarding pseudo-boiling in supercritical jet configurations. Jet decay and spreading rate are both enhanced compared to the other non-isothermal case, but may be even more so due to pseudo-boiling. A further investigation regarding lighter density supercritical injection could be useful in exploring this effect further. 

Some limitations of this work are now noted. \gls{sgs} models do not take into consideration real gas behavior, being limited to assumptions based on the ideal gas equation of state. There is room for improvement in turbulence modeling with the aim of incorporating higher order equations of state to the model formulation. As with any numerical simulation, discretization and resolution inherently lead to error. Ideally with more power, finer resolution can be achieved to help improve simulations. Finally, minor artifacts were noted in the non-isothermal case closest to the critical point likely due to the presence of steep gradients and issues with post-processing procedures. Though the impact of these are believed to be mild, they are still present and add to cumulative error. 

Future work aims to incorporate multiple species to the flow system and to investigate higher Mach flows. The impact of more specific application-oriented configurations with complex geometries on the flow field is also of interest. 


