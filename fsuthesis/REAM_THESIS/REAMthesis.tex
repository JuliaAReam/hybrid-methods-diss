%!TEX TS?program = pdflatexmk

% This is a "bare-bones" thesis template file.  For examples of how to
% use a few more LaTeX features, look in the 'sample' folder.  Read
% the User Guide for documentation of the 'fsuthesis' class features.
% Follow the FSU 'Guidelines and Requirements for Electronic Theses,
% Treatises, and Dissertations (ETDs)' document for all entries.

\documentclass[11pt,expanded,copyright]{fsuthesis}

% Additional packages may be loaded here.

\usepackage{amsmath, physics}
\usepackage{nicefrac}
\usepackage{siunitx}

\usepackage[acronyms]{glossaries}
\newacronym{co2}{CO2}{Carbon Dioxide}
\newacronym{sco2}{sCO2}{supercritical Carbon Dioxide}
\newacronym{les}{LES}{large eddy simulation}
\newacronym{fda}{FDA}{Food and Drug Administration}
\newacronym{dns}{DNS}{direct numerical simulation}
\newacronym{egs}{EGS}{Enhanced Geothermal Systems}
\newacronym{cfd}{CFD}{computational fluid dynamics}
\newacronym{rans}{RANS}{Reynolds-averaged Navier-Stokes}
\newacronym{sgs}{SGS}{subgrid-scale}
\newacronym{srk}{SRK EoS}{Soave-Redlich-Kwong Equation of State}
\newacronym{pr}{PR EoS}{Peng-Robinson Equation of State}
\newacronym{eos}{EoS}{Equation of State}
\newacronym{nist}{NIST}{National Institute of Standards and Technology}
\newacronym{hpc}{HPC}{high-performance computing}
\newacronym{smd}{SMD}{dynamic Smagorinsky}
\newacronym{ppm}{PPM}{piecewise parabolic method}
\newacronym{nrel}{NREL}{National Renewable Energy Laboratory}
\newacronym{rms}{rms}{root mean square}
\newacronym{amr}{AMR}{adaptive mesh refinement}
\newacronym{ecp}{ECP}{Exascale Computing Project}
\newacronym{doe}{DOE}{Department of Energy}
\newacronym{hmhw}{HMHW}{half-mean half-width}
\newacronym{tke}{TKE}{turbulent kinetic energy}
\newacronym{sdc}{SDC}{Spectral Deferred Correction}
\newacronym{mol}{MOL}{Method of Lines}
\newacronym{ode}{ODE}{ordinary differential equation}
\newacronym{pde}{PDE}{partial differential equation}
\newacronym{misdc}{MISDC}{multi-implicit spectral deferred correction}

\makeglossaries

\ifpdf   % We execute this part (up to \else) if we are in PDF mode.
  \usepackage[pdftex]{graphicx}
  % We can create color text in our document using the color package.
  % We load it here so that we can tweak the default colors used in
  % the links created by hyperref, as the defaults are a bit too
  % bright and gaudy for an austere document such as this. ;-)
  % Note that the Manuscript Clearance Advisors will not allow any
  % colored text at all, so if you're using hyperref, you'll need to
  % set all the colors to black (or set colorlinks=false below).
  \usepackage{xcolor}
  \definecolor{mygreen}{rgb}{0,0.6,0}
  \definecolor{myblue}{rgb}{0.3,0.2,0.8}
  \definecolor{myred}{rgb}{0.8,0.1,0.1}
  
  % Define some colors
\definecolor{c1med}{HTML}{F15A60} % red
\definecolor{c2med}{HTML}{7AC36A} % green
\definecolor{c3med}{HTML}{5A9BD4} % blue
\definecolor{c4med}{HTML}{FAA75B} % orange
\definecolor{c5med}{HTML}{9E67AB} % purple
\definecolor{c6med}{HTML}{CE7058} % burgundy
\definecolor{c7med}{HTML}{D77FB4} % magenta
\definecolor{c8med}{HTML}{737373} % grey

\definecolor{c1brt}{HTML}{EE2E2F} % red     
\definecolor{c2brt}{HTML}{008C48} % green   
\definecolor{c3brt}{HTML}{185AA9} % blue    
\definecolor{c4brt}{HTML}{F47D23} % orange  
\definecolor{c5brt}{HTML}{662C91} % purple  
\definecolor{c6brt}{HTML}{A21D21} % burgundy
\definecolor{c7brt}{HTML}{B43894} % magenta 
\definecolor{c8brt}{HTML}{010202} % black
  % The hyperref package has lots of features and options.  If you
  % are interested in creating a state-of-the-art PDF document out
  % of your thesis/dissertation, you should read more about this
  % extremely useful package.
 \usepackage[colorlinks=true,bookmarks=true,pdfborder={0 0 0},
   linkcolor=myblue,urlcolor=myred,citecolor=mygreen,
    breaklinks=true,bookmarksnumbered=true]{hyperref}
\else    % Otherwise we'll run this part if we are not in PDF mode.
  \usepackage[dvips]{graphicx}
\fi

\usepackage{tikz}
\usetikzlibrary{hobby,calc,decorations.markings,arrows.meta,patterns}
\usepackage{subcaption}
\usepackage{float}
\usepackage[utf8]{inputenc} % usually not needed (loaded by default)
\usepackage[T1]{fontenc}


% Do NOT use 'geometry', 'setspace', or 'tocloft' packages! They will
% mess up the spacing provided by 'fsuthesis'.

% If using the BibLaTeX package for generating a bibliography or
% references section, uncomment the following 5 lines. This is a
% basic starting point, and you may want other package options.
% You'll also need to uncomment the '\printbibliography' line
% toward the end of this file. Refer to the biblatex package
% documentation for information. The file 'myrefs.bib' contains
% a few sample references. 

\usepackage[american]{babel}
\addto\captionsamerican{\renewcommand*{\contentsname}{Table of Contents}}
\usepackage{csquotes}
\usepackage{biblatex}
\addbibresource{myrefs.bib}

\newcommand{\overbar}[1]{\mkern 1.5mu\overline{\mkern-1.5mu#1\mkern-1.5mu}\mkern 1.5mu}


%\usepackage{cite}

%
% Update these entries. See the User Guide for more information.
%
% \title{Supercritical Carbon Dioxide Round Turbulent Jets:\protect\\ Using Large Eddy Simulations to Investigate Fundamental Flow Physics of Complex Fluids}
\title{Large Eddy Simulations to Investigate the Fundamental Flow Physics of Supercritical Carbon Dioxide Turbulent Jets}
\author{Julia Ream}
\college{College of Arts \& Sciences}
\department{Department of Mathematics}  % Delete if no department
\manuscripttype{Dissertation}              % [Thesis, Dissertation, Treatise]
\degree{Doctor of Philosophy}               % [Master of Science, Doctor of Philosophy...]  
\degreeyear{2023}
\defensedate{July 11, 2023}

%
% If creating a PDF, you may want to uncomment these and enter
% appropriate metadata for your document. See the User Guide.
%
%\subject{My Topic}
%\keywords{keyterm1; keyterm2; keyterm3; ...}
%

%
% Update to use the real names and positions of each person.
% See FSU's 'Guidelines and Requirements' document for the proper
% formats and titles.
%
\committeeperson{Mark Sussman}{Professor Co-Directing Dissertation}
\committeeperson{Marc Henry de Frahan}{Professor Co-Directing Dissertation}
\committeeperson{Bryan Quaife}{University Representative}
\committeeperson{Aseel Farhat}{Committee Member}
\committeeperson{Sanghyun Lee}{Committee Member}

\clubpenalty=9999
\widowpenalty=9999

\begin{document}

\frontmatter
\maketitle
\makecommitteepage

\begin{dedication}
Fill in dedication later.
\end{dedication}

\begin{acknowledgments}
Fill in acknowledgments later.
\end{acknowledgments}

\tableofcontents
\listoftables
\listoffigures
%\listofmusex


%\begin{listofsymbols}
%\end{listofsymbols}

\begin{listofabbrevs}
The following short list of abbreviations are used throughout this document. 
\begin{center}
\begin{tabular}{ll}
AMR& Adaptive Mesh Refinement \\
CFD& Computational Fluid Dynamics \\
CO2& Carbon Dioxide \\
DNS& Direct Numerical Simulation \\
DOE& Department of Energy \\
ECP& Exascale Computing Project \\
EoS& equation of state \\
HMHW& half-mean half-width \\
HPC& high-performance computing \\
LES& Large Eddy Simulation \\
MISDC& Multi-Implicit Spectral Deferred Correction \\
MOL& Method of Lines \\
NIST& National Institute of Standards and Technology \\
NREL& National Renewable Energy Laboratory \\
ODE& Ordinary Differential Equation \\
PDE& Partial Differential Equation \\
PPM& piecewise parabolic method \\
PR EoS& Peng-Robinson equation of state \\
RANS& Reynolds-Averaged Navier-Stokes \\
rms& root mean square \\
SDC& Spectral Deferred Correction \\
sCO2& Supercritical Carbon Dioxide \\
SGS& Subgrid Scale \\
SMD& dynamic Smagorinsky \\
SRK EoS& Soave-Redlich-Kwong equation of state \\
TKE& Turbulent Kinetic Energy \\





\end{tabular}
\end{center}
\end{listofabbrevs}

\begin{abstract}
Supercritical carbon dioxide (sCO2) is of interest to a wide range of engineering problems, including carbon capture, utilization, and storage (CCUS) as well as advanced cycles for power generation. Non-ideal variations in physical properties of sCO2 impact the physics of these systems. In this study, we simulate turbulent sCO2 jets to gain a better understanding of these physics. Of particular interest is the impact of pseudo-boiling on supercritical flow dynamics. We use a second order finite volume method with adaptive mesh refinement as implemented in the first-principles simulation code PeleC to perform a Large Eddy Simulation (LES) of three turbulent jets of sCO2. Additionally, we use the Soave-Redlich-Kwong equation of state to close the system and examine the impact of a cubic equation of state on the turbulent flow physics. We look at velocity and Reynolds stress profiles at different downstream locations for three cases in which the temperature of the jet and that of the ambient fluid differ. These results are then contrasted with established theory for ideal gas jets and similar numerical studies involving transcritical injection in order to capture the effects of widely varying thermal properties in the pseudocritical region.
\end{abstract}

\mainmatter

\input Introduction
\input Model
\input Numerics
\input Simulations
\input Results
\input Conclusion


\appendix
\input Transport_Appendix
\input Thermo_Appendix
\input Inputs_Appendix

% You have your choice of bibliography sections: either
% hand-crafted or BibLaTeX. The hand-crafted References
% section is enabled by default. Remove these entries if
% you're using BibLaTeX.

% This is the "hand-crafted" bibliography/references section:
%\begin{references}
%Mybib, Sample. \textit{An Example of a Bibliographic Entry
 %Created Manually}. Tallahassee, Florida: Fornish and Frak, 2010.

%Smith, Marigold. \textit{Lots and Lots of Bibliographic Entries
 %and How to Display Them}. Tallahassee, Florida: Gibson and Goulash, 2010.
%\end{references}

% Or use the BibLaTeX bibliography package, uncomment the lines in the
% preamble, and uncomment the \printbibliography line below. View the
% file 'myrefs.bib' to get a feel for what these entries may look
% like. See the document in the 'sample' folder for more citation and
% BibLaTeX examples.

\printbibliography

%\bibliography{myrefs}{}
%\bibliographystyle{plain}

\begin{biosketch}
Julia Ream was born and raised in Sarasota, FL. From an early age, Julia found math to be ``fun", but didn't come to realize how uncommon that sentiment was until much later on in life. She started her undergraduate career at Florida State University in 2013, enrolling as an Exploratory student before committing to the Humanities and later on Theatre. Upon realizing that she missed doing math (weird), she added Applied Mathematics to her roster of majors. Although it took nearly 21 years, Julia eventually realized that she actually really liked math. So, after completing a Bachelor of Arts in Humanities and a Bachelor of Science in Mathematics, she decided to pursue her Doctoral degree in Applied and Computational Mathematics, also at Florida State University. 
\end{biosketch}

\end{document}
